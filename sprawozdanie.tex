
\documentclass[varwidth,12pt,a4paper]{article}

\usepackage[utf8]{inputenc}
\usepackage{csvsimple}
\usepackage{booktabs}
\usepackage[MeX]{polski}
\usepackage{graphicx}   %do rysunków
\usepackage{wrapfig}    %do rysunków otoczonych tekstem
\usepackage{color}      %do użycia podst. kolorów oraz zdefiniowanych kolorów 
\usepackage{amsmath}
\usepackage{float}
\usepackage{fancyhdr}
\usepackage{subcaption}
\pagestyle{fancy}

%do kolorowych referencji do rysunków, cytowań:
\usepackage{multicol}
\usepackage{colortbl}
\usepackage[colorlinks=true,linkcolor=firebrick,citecolor=green]{hyperref}

% paczki do ogarnięcia kodu
\usepackage{listings}
\usepackage{xcolor}

\definecolor{codegreen}{rgb}{0,0.6,0}
\definecolor{codegray}{rgb}{0.5,0.5,0.5}
\definecolor{codepurple}{rgb}{0.58,0,0.82}
\definecolor{backcolour}{rgb}{0.95,0.95,0.92}

\lstdefinestyle{mystyle}{
    backgroundcolor=\color{backcolour},   
    commentstyle=\color{codegreen},
    keywordstyle=\color{magenta},
    numberstyle=\tiny\color{codegray},
    stringstyle=\color{codepurple},
    basicstyle=\ttfamily\footnotesize,
    breakatwhitespace=false,         
    breaklines=true,                 
    captionpos=b,                    
    keepspaces=true,                 
    numbers=left,                    
    numbersep=5pt,                  
    showspaces=false,                
    showstringspaces=false,
    showtabs=false,                  
    tabsize=2
}

\lstset{style=mystyle}


%do zdefiniowana własnych kolorów:
\definecolor{darkred}{rgb}{0.5,0,0}
\definecolor{darkblue}{rgb}{0,0,0.5}
\definecolor{firebrick}{rgb}{0.75,0.125,0.125}
\definecolor{darkgreen}{rgb}{0,0.5,0}

\usepackage[
  top=2cm,
  bottom=1cm,
  left=3cm,
  right=2cm,
  headheight=17pt, % as per the warning by fancyhdr
  includehead,includefoot,
  heightrounded, % to avoid spurious underfull messages
]{geometry} 

\title{Sprawozdanie}
\author{Artur Dziedziczak}
\date{\today}


%%%%%%%%%%%%%%%%%%%%%%%%%%%%%%%%%%%%%%%%%%%%%%%%%%%%%
\begin{document}

\maketitle

\section{Opis zjawiska}

Treść zadania jest rozbudowana jednak na wstępie chciałbym powtórzyć opis zjawiska, które w dalszej części będzie podlegało symulacji.

Zjawiskiem tym jest proces chłodzenia metalowego pręta w specjalnym oleju. Do najważniejszych elementów zjawiska należą zmiany temepratury
pręta $T_p$ oraz oleju $T_w$.

Olej jak i pręt posiadają pewną masę $m_w, m_b$, od której w dużej mierze zależy szybkość zmian temperatury. Jeżeli 
duży kontener wypełnimy olejem i wrzucimy do niego pręt o małej masie zostanie on szybciej schłodzony.

Na szybkość zachodzenia reakcji zmian temperatury ma również wpływ współczynnik $A$, który jest sumaryczną powierzchnią
ścianek pręta.

Poszczególne zadania z instrukcji wymagają użycia metod numerycznych do wykonania symulacji procesu chłodzenia, który
od pewnego momentu nie odnosi się do stałego współczynnika przewodności cieplnej $h$ i wymagane jest użycie metod
interpolacji oraz aproksymacji tak aby uzależnić ten współczynnik od wyników eksperymentalnych.

Ostatnim zadaniem jest stworzenie programu, który pozwoli na zminimalizowanie kosztu schłodzenia $100000$ prętów
przy zmieniajacych się warunkach \textbf{ilości zbiorników, masy oleju oraz ilości napełnień zbiorników}.

\section{Model matematyczny}

Opisany w sekcji \textbf{Opis zjawiska} proces ma swój model matematyczny, który pozwala na określenie zmian temperatur.
\newline

Równania stanu

\begin{align}
    \frac{m_b c_b}{hA} \frac{dT_b}{dt} + T_b = T_w  \\
    \frac{m_w c_w}{hA} \frac{dT_w}{dt} + T_w = T_b  \nonumber
\end{align}

należy przekształcić do formy umożliwiającej obliczenie równań różniczkowych

\begin{equation}
     \begin{cases}
    \frac{\partial T_b}{\partial t}=\frac{(T_w-T_b)*h*a}{m_b*c_b} \\
    \frac{\partial T_w}{\partial t}=\frac{(T_b-T_w)*h*a}{m_w*c_w} 
    \end{cases}
\end{equation}

układ tych równań różniczkowych przy pomocy metody Eulera i Ulepszonej metody Eulera pozwala na obliczenie $T_b, T_w$ dla pewnego kroku.

\lstinputlisting[language=Octave, caption=Funkcja równań stanu]{f.m}

\subsection{Metoda Eulera}

Zakładając, że mamy jakąś funkcję $\frac{dy}{dt}' = f(t,y)$
zgodnie z definicją pochodnej możemy zapisać
\begin{equation}
   \frac{dy}{dt} \approx \frac{\Delta y}{\Delta t} = \frac{y_{i+1} - y_i}{t_{i+1} - t_i} = f(t_i, y_i)
\end{equation}
możemy przekształcić przez przemnożenie przez $\Delta t$

\begin{equation}
    \Delta y = \Delta t * f(t_i, y_i)
\end{equation}

Co pozwala nam na przybliżanie wartości pochodnej numerycznie.

\lstinputlisting[language=Octave, caption=Funkcja Eulera]{eulerr.m}

\subsection{Ulepszona metoda Eulera}

Różni się od podstowej tym, że iteracje odbywają się wzdłóż prostej o współczynniku
kierunkowym równym współczynnikowi stycznej do krzywej w punkcie oddalonym od 
$x_0$ o $\frac{h}{2}$.

\begin{equation}
    y_n = y_{n-1} + hf(x_{n-1} + \frac{h}{2}, y_{n-1} + \frac{h}{2} f(x_{n-1}, y_{n-1})
\end{equation}

\lstinputlisting[language=Octave, caption=Funkcja ulepszonej metody Eulera]{ulepszony_euler.m}

\subsection{Funkcje sklejane 1 stopnia}

Interpolacja ta polega na stworzeniu wielomianów stopnia pierwszego pomiędzy $ [[x^k, x^{k+1}] $ przedziałami.

Postać ogólna wielomianu interpolacyjnego wygląda następująco:

\begin{equation}
    y = \alpha + \beta * x
\end{equation}

A współczynniki $\alpha$ oraz $\beta$ uzyskuje się przez rozwiazanie układu równań:

$$
\left( \begin{array}{cc}
    1 & x_k \\
    1 & x_{k + 1} \\
\end{array} \right) \left( \begin{array}{c}
\alpha\\
\beta \\
\end{array} \right) = \left( \begin{array}{c}
    y_k \\
    y_{k + 1} \\
\end{array} \right) 
$$

\lstinputlisting[language=Octave, caption=Funkcja interpolująca funkcjami sklejanymi 1 stopnia]{interpolacja_funkcjami_sklejanymi.m}

\subsection{Funkcje sklejane 3 stopnia}

Interpolacja funkcjami sklejanymi 3 stopnia (Cubic spines interpolation)

Interpolacja ta odbywa się poprzez stworzenie funkcji wielomianów trzeciego stopnia $ p_k(x) $
dla różnych interwałów $ [[x^k, x^{k+1}] $.

$$
    f(x) = \left\{ \begin{array}{ll}
    p_1 (x) & \textrm{$x_1 <= x < x_2$}\\
    p_2 (x) & \textrm{$x_2 <= x < x_3$}\\
    \vdots  & \vdots \\
    p_{m-1} (x) & \textrm{$x_{m-1} <= x < x_m$}\\
    \end{array} \right.
$$

Dla punktów $(x_1, y_1), (x_2, y_2), ..., (x_m, y_m)$ dla, których $x_1 < x_2 < ... < x_m $
należy utworzyć krzywą sześcienną poprzez interpolacje wielomianu stopnia trzeciego $p_k$ 
pomiędzy każdą z kolejnych par punktów $(x_k, y_k)$ oraz $x_{k}, y_{k+1}$ zgodnie z takimi ograniczeniami:

1. Każdy z wielomianów przechodzi przez kolejne punkty
\begin{equation}
    p_k (x_k) = y_k oraz p_k (x_{k+1}) = y_{k + 1}
\end{equation}

2. Każda pierwsza pochodna dla punktów wewnętrznych jest sobie równa
\begin{equation}
     \frac{\partial d}{\partial dx} p_jk (x_{k+1}) = \frac{\partial d}{\partial dx} p_{k+1} (x_{k+1})
\end{equation}

3. Druga pochodna dla punktów wewnętrznych jest sobie równa
\begin{equation}
     \frac{\partial d^2}{\partial dx^2} p_jk (x_{k+1}) = \frac{\partial d^2}{\partial dx^2} p_{k+1} (x_{k+1})
\end{equation}

4. Druga pochodna wynosi zero na punktach końcowych
\begin{equation}
     \frac{\partial d^2}{\partial dx^2} p_1 (x_{1}) = 0 oraz \frac{\partial d^2}{\partial dx^2} p_{m-1} (x_{m}) = 0
\end{equation}

Wielomian, który rozpatrujemy ma postać
\begin{equation}
    p(x) = a_i \Delta x^3 + b_i \Delta x^2 + c_i \Delta x + d_i
\end{equation}
\begin{equation}
    p'(x) = 3 a_i \Delta x^2 + 2b \Delta x + c
\end{equation}
\begin{equation}
    p''(x) = 6 a_i \Delta x + 2b
\end{equation}

I te warunki pozwalają na ułożenie układu równań liniowych.

Zanim zaprogramowałem ten układ postawnowiłem rozwiązać sprawdzić jak taki układ powinien wyglądać dla 4 punktów.

Wybrane przeze mnie punkty to $(x_k, y_k)  = (1,1), (2,5), (3,4), (4, 9) $

Najpierw określam ile powinno być wielomianów w układzie równań 
\begin{equation}
n=(2*(N-2)+2)/2 = (2*(4-2)+2)/2 = 3 
\end{equation}
Możemy więc wyróżnić trzy wielomiany:

\begin{equation}
    p_1(x) = \delta _1 x^3 + \gamma _1 x^2 + \beta _1 x + \alpha _1
\end{equation}
\begin{equation}
    p_2(x) = \delta _2 x^3 + \gamma _2 x^2 + \beta _2 x + \alpha _2
\end{equation}
\begin{equation}
    p_3(x) = \delta _3 x^3 + \gamma _3 x^2 + \beta _3 x + \alpha _3
\end{equation}

Wektor niewiadomych, które musimy obliczyć wygląda następująco:
\begin{equation}
    [\delta _1, \delta_2, \delta_3, \gamma _1, \gamma _2, \gamma_3, \beta _1, \beta _2, \beta _3, \alpha _1, \alpha _2, \alpha _3 ]
\end{equation}


Teraz rozpatruje warunki:

1. Każdy z wielomianów przechodzi przez kolejne punkty
\begin{equation}
    p_1(1) = \delta _1 x_1^3 + \gamma _1 x_1^2 + \beta _1 x_1 + \alpha _1 = 1
\end{equation}
\begin{equation}
    p_1(2) = \delta _1 x_2^3 + \gamma _1 x_2^2 + \beta _1 x_2 + \alpha _1 = 5
\end{equation}
\begin{equation}
    p_1(2) = \delta _2 x_2^3 + \gamma _2 x_2^2 + \beta _2 x_2 + \alpha _2 = 5
\end{equation}
\begin{equation}
    p_1(3) = \delta _2 x_3^3 + \gamma _2 x_3^2 + \beta _2 x_3 + \alpha _2 = 4
\end{equation}
\begin{equation}
    p_1(3) = \delta _3 x_3^3 + \gamma _3 x_3^2 + \beta _3 x_3 + \alpha _3 = 4
\end{equation}
\begin{equation}
    p_1(4) = \delta _3 x_4^3 + \gamma _3 x_4^2 + \beta _3 x_4 + \alpha _3 = 9
\end{equation}

2. Każda pierwsza pochodna dla punktów wewnętrznych jest sobie równa
\begin{equation}
    3 \delta _1 x_2 ^2 + 2 \gamma _1 x_2 + \alpha _1 = 3 \delta _2 x_2 ^2 + 2 \gamma _2 x_2 + \alpha _2
\end{equation}
\begin{equation}
    3 \delta _2 x_3 ^2 + 2 \gamma _2 x_3 + \alpha _2 = 3 \delta _3 x_3 ^2 + 2 \gamma _3 x_3 + \alpha _3
\end{equation}

3. Druga pochodna dla punktów wewnętrznych jest sobie równa
\begin{equation}
    6\delta _1 x_2 + 2 \gamma _1 = 6 \delta _2 x_2 + 2 \gamma _2
\end{equation}
\begin{equation}
    6\delta _2 x_2 + 2 \gamma _2 = 6 \delta _3 x_3 + 2 \gamma _3
\end{equation}

4. Druga pochodna wynosi zero na punktach krańcowych
\begin{equation}
    6\delta _1 x_1 + 2 \gamma _1 = 0
\end{equation}
\begin{equation}
    6\delta _3 x_4 + 2 \gamma _3 = 0
\end{equation}

Przekształcam te równania w macierz tridiagonalną, która połączona jest z dodatkowymi warunkami pochodnych. 
W ten sposób całe obliczenie można wykonać używając operatora rozwiązania równań liniowych w Matlab.


$$
\left( \begin{array}{cccccccccccc}
     1  &  1  &  1  &  1 &   0&    0&    0 &   0&    0&    0&    0&    0 \\
     8  &  4  &  2  &  1 &   0&    0&    0 &   0&    0&    0&    0&    0 \\
     0  &  0  &  0  &  0 &   8&    4&    2 &   1&    0&    0&    0&    0 \\
     0  &  0  &  0  &  0 &  27&    9&    3 &   1&    0&    0&    0&    0 \\
     0  &  0  &  0  &  0 &   0&    0&    0 &   0&   27&    9&    3&    1 \\
     0  &  0  &  0  &  0 &   0&    0&    0 &   0&   64&   16&    4&    1 \\
    12  &  4  &  1  &  0 & -12&   -4&   -1 &   0&    0&    0&    0&    0 \\
     0  &  0  &  0  &  0 &  27&    6&    1 &   0&  -27&   -6&   -1&    0 \\
    12  &  2  &  0  &  0 & -12&   -2&    0 &   0&    0&    0&    0&    0 \\
     0  &  0  &  0  &  0 &  18&    2&    0 &   0&  -18&   -2&    0&    0 \\
     6  &  2  &  0  &  0 &   0&    0&    0 &   0&    0&    0&    0&    0 \\
     0  &  0  &  0  &  0 &   0&    0&    0 &   0&    0&   24&    8&    0 \\
\end{array} \right) \left( \begin{array}{c}
\delta _1\\
\delta_2\\
\delta_3\\
\gamma _1\\
\gamma _2\\
\gamma_3\\
\beta _1\\
\beta _2\\
\beta _3\\
\alpha _1\\
\alpha _2\\
\alpha _3 \\
\end{array} \right) = \left( \begin{array}{c}
1 \\
5 \\
5 \\
4 \\
4 \\
9 \\
0 \\
0 \\
0 \\
0 \\
0 \\
0 \\
\end{array} \right) 
$$

Z czego współczynniki po rozwiązaniu ukłądu równań wynoszą
$$
\left( \begin{array}{c}
\delta _1\\
\delta_2\\
\delta_3\\
\gamma _1\\
\gamma _2\\
\gamma_3\\
\beta _1\\
\beta _2\\
\beta _3\\
\alpha _1\\
\alpha _2\\
\alpha _3 \\
\end{array} \right) = \left( \begin{array}{c}
-1.1724 \\
3.5172 \\
1.6552 \\
-3.0000 \\
0.8621 \\
-8.6897 \\
26.0690 \\
-19.2759 \\
8.7241 \\
-79.4483 \\
238.3448 \\
-231.5517 \\
\end{array} \right) 
$$

\lstinputlisting[language=Octave, caption=Funkcja obliczająca współczynniki funkcji sklejanych 3 stopnia]{oblicz_wspolczynniki_fn_sklejanych_3_stopnia.m}

\lstinputlisting[language=Octave, caption=Funkcja interpolująca wielomian dla współczynników funkcji sklejanych 3 stopnia]{interpoluj_wspolczynniki_fn_3_stopnia.m}

\subsection{Metoda Simpsona}

Metoda pozwala na numeryczne przybliżenie całki za pomocą zastąpienia przedziałów całkowania łukiem paraboli.
\begin{equation}
\int_{x_0}^{x_n} f(x)dx = \sum_{k}^{i=1} \int_{2_{i-2}}^{x_{2i}} f(x)dx \approx \frac{h}{3}(y_0 + 4 \sum_{i=1}^{k} y_{2i-1} + 2 \sum_{i=1}^{k-1} y_{2i} + y_n)
\end{equation}

\lstinputlisting[language=Octave, caption=Funkcja obliczająca wartość całki metodą parabol]{calkowanie_numeryczne_parabol.m}

\subsection{Splajny równoodległych punktów}

Dla przedziały, który ma odległość $h = \frac{b - a}{n}$.

Otrzymuje się węzły równoodległe $x_i = a + i * h$ gdzie $i = 0,1,...,n$

Pomiędzy, którymi oblicza się wartości dla \textbf{funkcji bazowej}
opisanej wzorem:

\begin{equation}
  \Phi_{i} (x) = \frac{1}{h^3} *
    \begin{cases}
      (x - x_{i-2})^3 & \text{dla $x \in <x_{i-2}, x_{i-1}> $}\\
      (x - x_{i-2})^3 - 4(x-x_{i-1})^3 & \text{dla $x \in <x_{i-1}, x_{i}> $}\\
      (x_{i+2} - x)^3 - 4(x_{i+1} - x)^3 & \text{dla $x \in <x_{i}, x_{i_{i+1}}> $}\\
      (x_{i+2} - x)^3 & \text{dla $x \in <x_{i+1}, x_{i+2}> $}\\
      0 & \text{dla $x \in R - <x_{i-2}, x_{i+2}> $}\\
    \end{cases}       
\end{equation}

Za pomocą zbioru tych funkcji można określić \textbf{funkcję sklejaną 3-iego stopnia}

\begin{equation}
    S_3(x) = \sum_{i=-1}^{n+1} c_i \Phi_i (x)
\end{equation}

gdzie $c_i$ są to współczynniki, które będą dobierane tak aby splajn był
funkcją interpolacyjną.

Wyznaczenie tych współczynników odbywa się poprzez rozwiązanie macierzy $n + 1$
równań liniowych  o następującej postaci


$$
\left( \begin{array}{cccccccc}
    4c_0 & 2c_1 \\
    c_0 & 4c_1 & c_2 \\
     &  & 4c_2 && c_3 \\
     &  & \cdots && \cdots \\
     &  &  & & & c_{n-2} & 4c_{n-1} & c_n \\
     &  &  & & & & 2c_{n-1} & 4c_n \\
\end{array} \right) = \left( \begin{array}{c}
    y_0 + \frac{h}{3} * \alpha \\
    y_1 \\
    y_2 \\
    \cdots \\
    y_{n-1} \\
    y_n - \frac{h}{3} * \beta \\
\end{array} \right) 
$$

gdzie $S_3'(a^+) = \alpha $ oraz $S_3'(b^-) = \beta$

przy czym ten układ pozwala tylko na rozwiązanie współczynników $c_0, c_1,...,c_n$
dlatego pozostałe współczynniki oblicza się z wzorów

\begin{equation}
    c_{-1} = c_1  - \frac{h}{3} * \alpha \\
\end{equation}
\begin{equation}
    c_{n+1} = c_{n-1}  - \frac{h}{3} * \beta 
\end{equation}

\lstinputlisting[language=Octave, caption=Funkcja obliczająca splajny 3 stopnia dla równoodległych punktów]{splajny_proste.m}

\subsection{Metoda Newtona-Raphsona}

Metoda ta pozwala na numeryczne przybliżenie pierwiastka równania dowolnej ciągłej funkcji $f$.

Wzór do obliczenia tego pierwiastka można wyrazić rekurencyjnie:
\begin{equation}
    x_{k+1} = x_k - \frac{f(x_k)}{f'(x_k)}
\end{equation}

\section{Część 1}

Cele zadania pierwszego:

\begin{itemize}
  \item Zaprezentowanie przebiegów czasowych dla różnych wartości z tabeli
  \item Określenie błędów w odniesieniu do pomiarów z tabeli i przedyskutowanie wyników
  \item Pokazanie różnicy w dokładności metody Eulera i Ulepszonej Metody Eulera
\end{itemize}

\subsection{Wykresy dla wartości z tabeli}

Współczynniki, które zostały użyte do wygenerowania przebiegów czasowych podane są na wykresach.

\includegraphics[width=0.4\textwidth]{Pomiar_1_krok_100.png}\hspace{0.1\textwidth}%
\includegraphics[width=0.4\textwidth]{Pomiar_2_krok_100.png}\par
\includegraphics[width=0.4\textwidth]{Pomiar_3_krok_100.png}\hspace{0.1\textwidth}%
\includegraphics[width=0.4\textwidth]{Pomiar_4_krok_100.png}\par
\includegraphics[width=0.4\textwidth]{Pomiar_5_krok_100.png}\hspace{0.1\textwidth}%
\includegraphics[width=0.4\textwidth]{Pomiar_6_krok_100.png}\par
\includegraphics[width=0.4\textwidth]{Pomiar_7_krok_100.png}\hspace{0.1\textwidth}%
\includegraphics[width=0.4\textwidth]{Pomiar_8_krok_100.png}\par
\includegraphics[width=0.4\textwidth]{Pomiar_9_krok_100.png}\hspace{0.1\textwidth}%
\includegraphics[width=0.4\textwidth]{Pomiar_10_krok_100.png}\par

\subsection{Błędy w odniesieniu do danych eksperymentu}

\begin{table}[H]
    \centering\csvautobooktabular{zad1_dane_krok_100.csv}
    \caption{Tabela błędów względnych dla metod Eluera w porównaniu do wartości temperatur z eksperymentu}
    \label{table:tabelaPomiarowBledowEuler}
\end{table}

\begin{itemize}
    \item $Tb0$ - wartość początkowa temperatury pręta,
    \item $Tw0$ - wartość początkowa oleju,
    \item $czasy$ - czas zanurzenia w pręta w oleju,
    \item $Mw$ - wartość masy oleju,
    \item $Tbk$ - końcowa wartość temperatury pręta z eksperymentu,
    \item $Twk$ - końcowa wartość temperatury oleju z eksperymentu,
    \item $eETb$ - błąd metody Eulera dla ostatniej wartości temperatury pręta
    \item $eETw$ - błąd metody Eulera dla ostatniej wartości temperatury oleju
    \item $eUETb$ - błąd Ulepszonej metody Eulera dla ostatniej wartości temperatury pręta
    \item $eUETw$ - błąd Ulepszonej metody Eulera dla ostatniej wartości temperatury oleju
\end{itemize}

Zestawione w tabeli wartości temperatur końcowych są zbliżone do tych, które zostały wykazane w eksperymencie.
W zależności od użytego algorytmu \textbf{Eulera} \textbf{bląd względny} dla ostatnich wartości temperatur w większości
przypadków jest mniejszy od $2 [^\circ C]$. Nie jest to jednak duża różnica gdyż spektrum temperatur dla, których
prowadzone są symulacje mieści się w $25-1200 [^\circ C]$.

Obserwacje te pozwalają wysnuć wniosek \textbf{Wykorzystanie metod Eulera do symulacji zjawiska chłodzenia
pręta w oleju pozwala na stworzenie symulacji, która dostatecznie przybliża eksperymenty pomiarowe}.


\subsection{Porównanie metod Eulera}

\begin{figure}
    \includegraphics[width=0.8\textwidth]{Pomiar_1_krok_1.png}
    \caption{Wykres funkcji Eulera dla małej wartości kroku}
    \label{fig:wykresEulerMalyKrok}
\end{figure}
\begin{figure}
    \includegraphics[width=0.8\textwidth]{Pomiar_2_krok_100.png} 
    \caption{Wykres funkcji Eulera dla dużej wartości kroku}
    \label{fig:wykresEulerDuzyKrok}
\end{figure}

Aby porównać obie metody należy odwołać się nie tylko do wykresów i kroku czasu jaki został użyty ale również
do błędów względnych, które obliczone zostały w poprzedniej tabeli ~\ref{table:tabelaPomiarowBledowEuler}

Zacznę więc od porównania błędów obu metod do wartości temperatur z eksperymentu.

Dla dowolnego eksperymentu wartości blędów $eEUTw, eEUTb$ są mniejsze od $eETb, eETw$ z czego można wysnuć wniosek,
że \textbf{Ulepszona metoda Eulera lepiej przybliża rozwiązania równań różniczkowych niż metoda Eulera}.

Drugim elementem, na który zwróciłem uwagę jest różne zachowanie obu metod dla mniejszych wartości $\Delta t$.
Metoda Eulera mniej dokładnie przybliża rozwiązania gdy krok ich obliczania jest większy. Można to 
zaobserwować na wykresach ~\ref{fig:wykresEulerMalyKrok} oraz ~\ref{fig:wykresEulerDuzyKrok}.
Na wykresie z małym krokiem obie metody dobrze radzą sobie z przybliżaniem wartości jednak gdy
krok zostaje zwiększony wykres metody Eulera nie pokrywa się już z wykresem ulepszonej metody Eulera przy czym
wykres ulepszonej metody pozostaje bez zmian.

To pozwala mi wysnuć wniosek, że \textbf{Ulepszona metoda Eulera bardziej nadaje się do symulowania zjawiska chłodzenia
pręta w oleju niż metoda Eulera}.

\section{Część 2}

Cele zadania drugiego:

\begin{itemize}
  \item Wykonać aproksymacje metodą najmniejszych kwadratów
  \item Wykonać interpolacje funkcjami pierwszego stopnia
  \item Wykonać interpolacje funkcjami trzeciego stopnia
  \item Obliczyć całkę metodą Simpsona dla aproksymacji i funkcji sklejanych
  \item Sprawdzić wpływ modelu aproksymującego charakterystykę na wyniki symulacji
\end{itemize}

\subsection{Aproksymacja metodą najmniejszych kwadratów}

\begin{figure}[H]
    \includegraphics[width=\textwidth]{Stopnie_wielomianu_5.png} 
    \caption{Wykres aproksymacji dla stopni wielomianów $\le 5$}
\end{figure}

\begin{table}[H]
    \centering\csvautobooktabular{bledy_stopni_wielomianow_8.csv}
    \caption{Wartości błędów średnio kwadratowych dla punktów pomiarowych oraz wielomianów $p(x)$ gdzie $x$
    jest stopniem wielomianu.}
    \label{table:bledyAproksymacjiKwadratowej}
\end{table}

\begin{figure}[H]
    \includegraphics[width=\textwidth]{Stopnie_wielomianu_8.png} 
    \caption{Wykres aproksymacji dla większych stopni wielomianów.}
    \label{fig:duzeStopnieWielomianow}
\end{figure}

Do dalszych eksperymentów będę używał wielomianu stopnia 5 ponieważ 
mimo iż błąd dla poszczególnych punktów z charakterystyki jest mniejszy ~\ref{table:bledyAproksymacjiKwadratowej} to 
z wykresów wielomianów stopnia większego niż 5 ~\ref{fig:duzeStopnieWielomianow} wynika iż zupełnie nie przybliżają
one charakterystyki współczynnika $h$.

Z czego można wysnuć wniosek, że \textbf{Wyższy stopień wielomianu nie koniecznie oznacza lepsze przybliżenie funkcji}.

\subsection{Interpolacja funkcjami sklejanymi}

\begin{figure}[H]
    \includegraphics[width=\textwidth]{Charakterystyka_ruchomego_h.png} 
    \caption{Porównanie interpolacji oraz aproksymacji w przybliżaniu funkcji zmiany współczynnika cieplnego}
    \label{fig:interpolacjaAproksymacja}
\end{figure}

\begin{figure}[H]
    \centering\includegraphics[width=0.6\textwidth]{Rownoodlegle.png} 
    \caption{Interpolacja dla równoodległych punktów, które najpierw aproksymowane są
    metodą najmniejszych kwadratów}
    \label{fig:interpolacjaRowne}
\end{figure}

Z wykresu ~\ref{fig:interpolacjaAproksymacja} wynika, że funkcje sklejane zdecydowanie lepiej radzą sobie w przybliżeniu charakterystyki
współczynnika $h$. O ile \textbf{funkcje sklejane 1 stopnia} przybliżają charakterystykę poprzez funkcje
liniowe, które tworzą proste pomiędzy punktami to \textbf{funkcje sklejane 3 stopnia} tworzą idealnie
wygładzony wykres w punktach łączenia.

Dodatkowo nawet bez liczenia błędów pomiedzy aproksymacją i interpolacją sam wykres pokazuje, że błędy te
dla poszczególnych punktów eksperymentalnych byłyby mniejsze dla interpolacji gdyż funkcja interpolująca musi 
przechodzić przez te punkty. Tzn. w miejscach dla, których wykonuje się interpolacje błąd powinien być równy $0$.

Co prowadzi do wniosku \textbf{Interpolacja funkcjami sklejanymi lepiej przbliża funkcję zmian współczynnika cieplnego}.

Odnoszę jednak wrażenie, że nie koniecznie interpolacja dla tego rozwiązywanego w zadaniu problemu nie musi być najlepszą metodą.
Można by taki wniosek wysnuć gdyby ilość punktów pomiarowych była większa i wtedy dla grupy takich pomiarów
trzebaby obliczyć błędy pomiędzy wartością funkcji a każdym z pomiarów. Przede wszystkim zbiór danych, na których
wykonywane są metody interpolacji i aproksymacji jest bardzo ograniczony i oprócz gęstszego rozkładu w punkcie
$0 [^\circ C] dla \Delta T$ dla pozostałej charakterystyki zmian $h$ punkty pomiarowe umieszczone są bardzo rzadko.

Funkcje sklejane 3 stopnia z algorytmem dla obliczeń punktów równoodległych ~\ref{fig:interpolacjaRowne} są najgorszą metodą
jaką można zastosować do przybliżenia wartości współczynnika $h$.
Przede wszystkim obecny w zadaniu zbiór danych nie pozwala na użycie tej metody
bez uprzedniego wygenerowania równoodległych punktów za pomocą interpolacji lub aproksymacji.

Lepiej jest już użyć ogólnego wzoru na funkcje sklejane, który jest bardziej skomplikowany
ale lepiej przybliża wartość funkcji bez potrzeby generowania równoodległych punktów.

\subsection{Całka metodą Simpsona}

\begin{figure}[H]
    \includegraphics[width=\textwidth]{Metoda_simpsona.png} 
    \caption{Wykres pola pomiędzy funkcją aproksymującą dla stopnia wielomianu 5 oraz funkcji sklejanych}
    \label{fig:calkaSimpsona}
\end{figure}

Określenie całki pomiędzy tymi funkcjami nie doprowadziło mnie do żadnego wniosku.

\subsection{Wpływ modelu aproksymującego charakterystykę na wyniki}

\includegraphics[width=0.4\textwidth]{modele_aproksymujace_1.png}\hspace{0.1\textwidth}%
\includegraphics[width=0.4\textwidth]{modele_aproksymujace_2.png}\par
\includegraphics[width=0.4\textwidth]{modele_aproksymujace_3.png}\hspace{0.1\textwidth}%
\includegraphics[width=0.4\textwidth]{modele_aproksymujace_4.png}\par
\includegraphics[width=0.4\textwidth]{modele_aproksymujace_5.png}\hspace{0.1\textwidth}%
\includegraphics[width=0.4\textwidth]{modele_aproksymujace_6.png}\par
\includegraphics[width=0.4\textwidth]{modele_aproksymujace_7.png}\hspace{0.1\textwidth}%
\includegraphics[width=0.4\textwidth]{modele_aproksymujace_8.png}\par

\subsection{Podsumowanie}

Przedstawione w tej sekcji wykresy pokazują, że współczynnik $h$ nie ma dużego wpływu na 
wyniki symulacji rozkładu temperatury. W większości wykresów niezależnie od wartości współczynnika $h$
temperatury oleju oraz prętów pokrywają się.

Co prowadzi do wniosku \textbf{Współczynnik $h$ ma nieznaczny wpływ na charakterystykę rozkładu temperatury}
oraz \textbf{Wybranie metody do przybliżenia $h$ nie ma znaczenia dla przebiegów charakterystyki rozkładu temperatury}
przez co do pozostałych zadań można wybrać dowolną funkcję przybliżającą współczynnik $h$.
W moim przypadku wybrałem metodę aproksymacji.

\section{Część 3}

Cele zadania trzeciego:

\begin{itemize}
  \item Określić jak dużo oleju potrzeba do schłodzenia pręta w określonym czasie
  \item Stworzenie wykresu dla metody Netwona-Raphsona
\end{itemize}

\includegraphics[width=\textwidth]{NewtonRaphson.png} 

\subsection{Podsumowanie}

Rozwiązanie części 3 nie doprowadziło mnie do żadnych wniosków.

\section{Część 4}

\subsection{Opis rozwiązania}

Problem przedstwiony w zadaniu jest złożony i zawiera wiele możliwych rozwiązań, które w mniejszym
lub większym stopniu pozwolą określić minimalne koszty produkcji przy zachowaniu narzuconych
ograniczeń.

W swoim rozwiązaniu skupiłem się na wykonaniu symulacji przez 24 godziny dla wielu 
równocześnie chłodzących zbiorników a następnie za pomocą metody bisekcji zmniejszam
ilość wymaganych zbiorników oraz masę oleju, która ma wpływ na koszt tak aby poprzedni koszt
całego procesu chłodzenia dla 100 000 prętów zmniejszał się. 

Warunkiem końcowym tego zmniejszania jest moment gdy poprzedni koszt wykonania chłodzenia
prętów jest mniejszy od \textbf{100 PLN}. Wartość ta może zostać zmodyfikowana ale dla
zmniejszenia czasu wykonywania się poszczególnych obliczeń dla 24 godzin uznałem, że jest
ona wystarczająca jednak nie potrafię określić w jaki sposób dobrać ją bardziej metodologicznie.

\begin{table}[H]
    \centering\csvautobooktabular{wspolczynniki_symulacji.csv}
    \caption{Współczynniki użyte podczas symulacji}
\end{table}


\subsection{Wyniki}

Na poniższych wykresach znajduje się stan symulacji dla numeru bisekcji.

\begin{wrapfigure}{r}{0.5\textwidth}
  \vspace{-20pt}
  \begin{center}
    \includegraphics[width=0.48\textwidth]{Bisekcja.png}
  \end{center}
  \vspace{-20pt}
  \caption{Koszt chłodzenia prętów}
  \vspace{30pt}
  \label{fig:minKoszt}
\end{wrapfigure}

Wykres \ref{fig:minKoszt} prezentuje koszt dla poszczególnych bisekcji. Wykres oprócz bisekcji $2$
zmierza do minimum kosztu. Skok pomiędzy bisekcjami $1$ oraz $2$ spowodowany jest tym, że dla zmiany
ilości zbiorników widocznej na wykresie \ref{fig:minIloscZbiornikow} nie udało się uzyskać liczby
100 000 prętów dla zadanej konfiguracji $6$ zbiorników. W sytuacji gdy symulator nie jest w stanie
dla danej ilości zbiorników schłodzić zadanej ilości prętów to ilość zbiorników jest ona zwiększana o $1$ i cały
proces jest powtarzany. 

Tak naprawdę od momentu otrzymania wyniku $7$ zbiorników jako minimalnej ilości, która pozwala na
schłodzenie 100 000 prętów symulator stara się minimalizować masę oleju gdyż przedziały bisekcji
dla ilości zbiorników wyglądają następująco $a = 6$ oraz $b = 8$ z czego $\frac{b + a}{2} = 7$

\begin{wrapfigure}{l}{0.5\textwidth}
  \vspace{-20pt}
  \begin{center}
    \includegraphics[width=0.48\textwidth]{MasaOleju.png}
  \end{center}
  \vspace{-20pt}
  \caption{Masa oleju}
  \vspace{30pt}
  \label{fig:minMasaOleju}
\end{wrapfigure}

Wykres ~\ref{fig:minMasaOleju} obrazuje minimalizacje ilości oleju pomiędzy każdą bisekcją.
Interesującym momentem jest bisekcja pomiędzy $1$ a $2$ gdzie jest ona taka sama. Symulator w momencie 
gdy ilość zbiorników nie pozwala na schłodzenie 100 000 zbiorników zmienia ich ilość a samą masę
zostawia ponieważ masa z poprzedniej bisekcji również nie pozwoliła na schłodzenie dostatecznej
ilości prętów.
\newpage

\afterpage{\clearpage}

\begin{wrapfigure}{l}{0.5\textwidth}
  \begin{center}
    \includegraphics[width=0.48\textwidth]{PretyNaDobe.png}
  \end{center}
  \caption{Ilość prętów na dobę}
  \label{fig:minPretyNaDobe}
\end{wrapfigure}

Wykres ~\ref{fig:minPretyNaDobe} wskazuje jak zmieniała się ilość prętów dla poszczególnych bisekcji.
Duże zmiany w ilości prętów są spowodowane zmianą ilości zbiorników. Jak widać po $2$ bisekcji 
ilość prętów schłodzonych w 24h powoli spada i jest to spowodowane tym, że symulator minimalizuje koszt
jedynie przez zmianę masy oleju. 

\begin{figure}
    \centering
    \begin{subfigure}[b]{0.4\textwidth}
        \includegraphics[width=\textwidth]{IloscZbiornikow.png}
        \caption{Ilość zbiorników}
        \label{fig:minIloscZbiornikow}
    \end{subfigure}
    ~ %add desired spacing between images, e. g. ~, \quad, \qquad, \hfill etc. 
      %(or a blank line to force the subfigure onto a new line)
    \begin{subfigure}[b]{0.4\textwidth}
        \includegraphics[width=\textwidth]{IloscNapelnienZbiornikow.png}
        \caption{Ilość napełnień zbiornika}
        \label{fig:minIloscNapelnien}
    \end{subfigure}
    ~ %add desired spacing between images, e. g. ~, \quad, \qquad, \hfill etc. 
    %(or a blank line to force the subfigure onto a new line)
    \caption{Stany zbiorników}
\end{figure}

Ilość zbiorników, które pozwalają na schłodzenie 100 000 prętów to $8$. 

Wykres ~\ref{fig:minIloscNapelnien} pokazuje ile razy trzeba było napełnić wszystkie zbiorniki ponownie.
Ponowne napełnienie odbywa się gdy zbiornik schłodzi w jedym oleju $2000$ prętów.
Każde napełnienie dodawane jest do sumarycznego kosztu.

\afterpage{\clearpage}

\begin{table}[H]
    \hspace{-40pt}
    \centering\csvautobooktabular{symulator.csv}
    \caption{Wartości symulacji dla bisekcji}
\end{table}

\subsection{Podsumowanie}

Wykorzystane przeze mnie rozwiązanie jest dalekie od ideału. Gdybym miał większą wiedzę z zakresu
matematyki na pewno użyłbym innej techniki, która pozwoliłaby minimalizować większą ilość zmiennych
jednocześnie. Wykorzystana przeze mnie metoda bisekcji pozwala na znalezienie rozwiązania jednak
nie jest ona dobrym wyborem do minimalizacji kosztu dla dwóch zmiennych \textbf{masy oleju} oraz
\textbf{ilości prętów}.

Podsumowując najmniejszy koszt wyprodukowania 100 000 prętów można otrzymać przy $8$ równocześnie
pracujących zbiornikach, które mają masę wody $0.1543$ i napełnieniu tych zbiorników $56$ razy.

\section{Wnioski}

\begin{itemize}
    \item \textbf{Wykorzystanie metod Eulera do symulacji zjawiska chłodzenia pręta w oleju pozwala na stworzenie symulacji, która dostatecznie przybliża eksperymenty pomiarowe}
    \item \textbf{Ulepszona metoda Eulera lepiej przybliża rozwiązania równań różniczkowych niż metoda Eulera}
    \item \textbf{Ulepszona metoda Eulera bardziej nadaje się do symulowania zjawiska chłodzenia pręta w oleju niż metoda Eulera}.
    \item \textbf{Wyższy stopień wielomianu nie koniecznie oznacza lepsze przybliżenie funkcji}
    \item \textbf{Interpolacja funkcjami sklejanymi lepiej przbliża funkcję zmian współczynnika cieplnego}
    \item \textbf{Współczynnik $h$ ma nieznaczny wpływ na charakterystykę rozkładu temperatury}
    \item \textbf{Wybranie metody do przybliżenia $h$ nie ma znaczenia dla przebiegów charakterystyki rozkładu temperatury}
\end{itemize}

\section{Kod zadania 4}
Program podzielony jest na następujące pliki:

\lstinputlisting[language=Octave, caption=Częśc symulacyjna zawierająca zmienne globalne współczynników i generująca wykresy.]{symulator.m}
\lstinputlisting[language=Octave, caption=Klasa przechowująca stan zbiorników oraz stany prętów.]{Zbiorniki.m}
\lstinputlisting[language=Octave, caption=Klasa określająca stan w jakim znajduje się pojedynczy zbiornik.]{Zbiornik.m}

\section{Kod zadań 1-3}
\lstinputlisting[language=Octave, caption=Program generujący wykresy i tabele do zadań 1-3]{zad1.m}


\end{document}