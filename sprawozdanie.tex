
\documentclass[12pt,a4paper]{article}

\usepackage[utf8]{inputenc}
\usepackage[MeX]{polski}
\usepackage{graphicx}   %do rysunków
\usepackage{wrapfig}    %do rysunków otoczonych tekstem
\usepackage{color}      %do użycia podst. kolorów oraz zdefiniowanych kolorów 

%do kolorowych referencji do rysunków, cytowań:
\usepackage{multicol}
\usepackage{colortbl}
\usepackage[colorlinks=true,linkcolor=firebrick,citecolor=green]{hyperref}

%do zdefiniowana własnych kolorów:
\definecolor{darkred}{rgb}{0.5,0,0}
\definecolor{darkblue}{rgb}{0,0,0.5}
\definecolor{firebrick}{rgb}{0.75,0.125,0.125}
\definecolor{darkgreen}{rgb}{0,0.5,0}

\textwidth=16cm
\textheight=23cm
\topmargin=-2cm
\oddsidemargin=0cm
%%%% https://www.sharelatex.com/learn/Page_size_and_margins


\title{Sprawozdanie}
\author{Artur Dziedziczak}
\date{\today}


%%%%%%%%%%%%%%%%%%%%%%%%%%%%%%%%%%%%%%%%%%%%%%%%%%%%%
\begin{document}

\maketitle

\begin{abstract}
W dokumencie pokazane są podstawowe cele badania zderzeń ciężkich relatywistycznych jonów. W szczególności opisany jest diagram fazowy oraz omówione są plany poszukiwania punktu krytycznego silnie oddziałującej materii.     
\end{abstract}


%%%%%%%%%%%%%%%%%%%%%%%%%%%%%%%%%%%%%%%%%
\section{Wzory}

\begin{equation}
    \frac{\partial T_b}{\partial t}=\frac{(T_w-T_b)*h*a}{m_b*c_b}
\end{equation}

\begin{equation}
    \frac{\partial T_w}{\partial t}=\frac{(T_w-T_w)*h*a}{m_w*c_w}
\end{equation}

\end{document}

